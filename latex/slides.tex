\documentclass[12pt]{beamer}

\author{Jake Humphrey}
\title{A Stochastic Computational Approach for Accurate and Efficient Reliability Evaluation}
\subtitle{A Python Implementation}
\institute{Department of Electronic and Electrical Engineering\\
  Imperial College London\\
  \texttt{jbh111@ic.ac.uk}
}

\begin{document}

\begin{frame}[plain]
  \titlepage
\end{frame}

\begin{frame}{Reliability of Circuits}
Gates in a logic circuit are, alas, not perfect. They are susceptible to error, of which there are three main types:
\begin{itemize}
\item \textbf{Stuck-At-One Error:} The output of the gate goes high, regardless of the expected output.
\item \textbf{Stuck-At-Zero Error:} The output of the gate goes low, regardless of the expected output.
\item \textbf{Von Neumann Error:} The output of the gate becomes the inverse of the expected output.
\end{itemize}
\end{frame}

\begin{frame}{Masking Effects}
However, there is a chance that errors in one gate will not propagate all the way to an output. This could be due to one of the following \emph{masking effect}s
\begin{itemize}
\item \textbf{Electrical Masking:} The error does not have a large enough effect on the amplitude of the logic signal to be detected at an input.
\item \textbf{Temporal Masking:} The error is input to a latch but occurs at some point in time outside of the latch's detection window.
\item \textbf{Logical Masking:} The error does not pass through a multi-input logic gate because the value of the other input(s) fix(es) the output of the gate.
\end{itemize}
\end{frame}
\end{document}